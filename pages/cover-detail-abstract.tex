% 封面页

% 设置当前页面布局无页眉页脚
\pagestyle{cover}
\thispagestyle{cover}

% 单独设置封面页的页边距
\newgeometry{left=2.5cm,right=2.5cm,top=2cm,bottom=2cm}

% 请注意填写以下信息
\begin{center}
    \zihao{-4} \songti
    \begin{tabularx}{\textwidth}{l p{3.5cm} >{\raggedleft}X p{3.5cm}}
        分类号:           & \uline{\hfill 分类号 \hfill}   &
        单位代码:         & \uline{\hfill 10033 \hfill}      \\
        密{\quad}级:      & \uline{\hfill 一般为空 \hfill} &
        学{\quad\quad}号: & \uline{\hfill 学号 \hfill}
    \end{tabularx}
\end{center}

\vspace{16pt}

\begin{center}
    \includegraphics[width=0.3\paperwidth]{logo/name.png}
\end{center}

\vspace{14pt}

\linespread{1}

\begin{center}
    \zihao{-1} \songti
    % 如有需求,请更改
    硕士学位论文
    \\
    详细摘要
\end{center}

\linespread{1.5}

\vspace{30pt}

\begin{center}
    \includegraphics[width=0.15\paperwidth]{logo/cuc.png}
\end{center}

% 垂直间距,如有需要,酌情修改
\vspace{32pt}

\begin{center}
    \zihao{-2} \bfseries
    \begin{tabularx}{1.0\textwidth}{>{\songti}l X<{\centering }}
        % 如果标题过长,请模仿下文中格式拓展行数
        中文论文题目: & \uline{\hfill \songti 基于LaTex的硕士研究生 \hfill}           \\
                       & \uline{\hfill \songti 毕业论文模板的设计与实现 \hfill}        \\
        \\
        英文论文题目: & \uline{\hfill A Thesis Template Based On \hfill}      \\
                       & \uline{\hfill LaTex Design and Implementation \hfill}
    \end{tabularx}
\end{center}

% 垂直间距,如有需要,酌情修改
\vspace{20pt}

% 注意手动填写以下信息
\begin{center}
    \zihao{4} \songti
    \begin{tabularx}{.6\textwidth}{>{\songti}l >{\songti}X<{\centering}}
        申请人姓名: & \bfseries \uline{\hfill 张三 \hfill}        \\
        指导教师:   & \bfseries \uline{\hfill 李四 \hfill} \hfill \\
        专业名称:   & \uline{\hfill 王麻子 \hfill}                \\
        研究方向:   & \uline{\hfill 研究方向 \hfill}              \\
        所在学院:   & \uline{\hfill 新媒体研究院 \hfill}          \\
    \end{tabularx}
\end{center}

% 垂直间距,如有需要,酌情修改
\vspace{20pt}

\begin{center}
    \zihao{-3} \bfseries
    \begin{tabularx}{.5\textwidth}{>{\songti}l >{\songti}X<{\centering}}
        论文提交日期 & \uline{\hfill 202X年6月 \hfill}
    \end{tabularx}
\end{center}

\restoregeometry
